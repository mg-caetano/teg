\section{Motivación y Problema}

Actualmente, el manejador de wikis MediaWiki ofrece la posibilidad de consultar el historial de un artículo mediante una herramienta textual que permite visualizar los metadatos asociados a los cambios que ha sufrido un artículo (versiones). Adicionalmente, el historial de MediaWiki permite realizar comparaciones entre versiones y, si es necesario, revertir cambios.

Según Scalise \cite{Sca08}, los sistemas como MediaWiki no explotan propiedades presentes en los datos de los historiales; en particular, la evolución de cada artículo, la comunidad de usuarios participantes y la frecuencia de los cambios. Además enfatiza que carecen de una visualización gráfica de los datos. Por ello planteó como solución el uso de técnicas de ingeniería de modelos para incorporar reportes alternos y visualizaciones sobre estos historiales. Esta solución dividió el proceso en tres actividades: extracción, cálculo de propiedades y visualización.

Una implementación del proceso propuesto por Scalise, con énfasis en la actividad de extracción, fue elaborada por D'Apuzzo y Worwa \cite{Dap12} en la cual se creó una aplicación que permite recuperar historiales de artículos hospedados en MediaWiki (Web Scraper) acompañado de un Front-End sencillo para la visualización. En su estado actual, este Front-End permite visualizar un conjunto limitado de propiedades generales y es poco flexible en cuanto a las distintas gráficas que se pueden obtener de un mismo historial.
En este trabajo se propone superar estas limitaciones extendiendo el conjunto de métricas disponibles (incluyendo métricas específicas) y ofreciendo una gama de visualizaciones completa que incluya gráficos 2D y 3D.

\section{Objetivos del Trabajo}

El objetivo de este trabajo especial de grado es desarrollar un Front-End que permita visualizar propiedades asociadas a los historiales de artículos en Wikipedia.

Los objetivos específicos son:

\begin{itemize}
  \item Analizar y estudiar la aplicación WikimetricsUCV en su versión actual.
  \item Explorar el estado del arte en técnicas de visualización de datos.
  \item Escoger las tecnologías apropiadas para implementar una gama de visualizaciones flexibles e interactivas.
  \item Definir nuevas propiedades que puedan ser obtenidas a partir de los datos del historial de los artículos, haciendo uso del enfoque dirigido por ingeniería de modelos planteado Scalise.
  \item Definir una gama de visualizaciones flexibles e interactivas que permitan la detección de patrones, el análisis de la información a distintos niveles y se apoyen en el uso de una codificación visual efectiva.
  \item Diseñar e implementar un nuevo Front-End para la aplicación WikimetricsUCV.
\end{itemize}

\section{Estrategia de Solución}

\section{Tecnologías a Utilizar}

\section{Planificación Inicial del Proyecto}
